\documentclass[nobib]{tufte-handout}
\usepackage{amsmath,amssymb,amsthm}


\title{MATH260 --- Discrete Mathematics \\  Problem Set I}
\date{Fall 2017}

\begin{document}
\maketitle

Be certain you understand the problem set grading rubric\sidenote{See \url{https://jlmayfield.github.io/teaching/MATH260/}} before getting too far into the work.  Remember, you are not required to do all of these and some will be done in class. That being said, doing them all will certainly prepare you well for exams if for no other reason than some will be on the exam. It's also worth reminding you that solutions to the odd numbered problems can be found in the back of the book. 

\vspace{.25in}
\begin{tabular}{lll}
\underline{Chapter.Section} & \underline{Problems} & \underline{Page} \\
1.1 & 2,4,8,12,14,16,18,20,24,26,28,30,32,36,40,46,48,52 & 7  \\
1.2 & 2\{a,b,d,e,g\},6,8,10,16,14 & 10 \\
1.3 & 2,4,5,8,10,16& 14 \\
1.4 & 2,6,10,14,16,18 & 16 \\
1.5 & 2\{all\},4\{all\},6,8,10   & 18--19 \\
1.6 & 2\{all\},6 & 20 \\
1.7 & 2,4,6,10,12,14 & 23 \\
1.8 & 2,6\{a,b\},10\{a,b\},12,14 & 28
\end{tabular}

\end{document}
