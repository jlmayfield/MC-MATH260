\documentclass[nobib]{tufte-handout}
\usepackage{amsmath,amssymb,amsthm}


\title{MATH260 --- Discrete Mathematics \\  Problem Set Rubric}

\begin{document}
\maketitle

Each problem set is graded on a three point scale.  One point comes from doing a meta-cognitive analysis of the problem set, a half point comes from attempting a sufficient sample of the problems, and the remaining one and one half points come from the correctness and quality of the problems attempted.

\subsection*{Analysis}

The goal of the meta-cognitive analysis is to look at the problems assigned, evaluate their purpose, and place them in the wider context of the material from the chapter and the goals of the course. A good analysis should connect the specific problems to more general ideas and techniques presented in the chapter and the course and seek to clarify the wider applicability of those techniques and ideas. In doing this, your analysis should identify a minimal sub-set of the problem set such that doing those problem sufficiently covers the core ideas and skills covered in the relevant material.  In addition to listing the set of core problems, list any problems that you want reviewed in class because you're unsure of your solution as well as problems you want reviewed because you didn't even know where to start\sidenote{that's three sets of problems: core, unsure, and lost}.


The real goal here is to develop your ability to see transferable skills and processes within the specifics of each practice problem. Use no more than a page (\textit{typed}) to present your analysis. Alternatively, you can create a concept map or some other visualization to capture the results of your analysis.

\subsection*{Problems Attempted}

In the course of your meta-cognitive analysis you should identify a core set of problems from the problem set.  To get full credit for attempting problems you must attempt these problems and these problems should, in fact, be a good representative sample of problems for the chapter. Alternatively, you can take the shotgun approach and just do all the problems.

\subsection*{Correctness and Quality}

Full credit for the problem set requires that you meet some level of success on the problems you attempt and that the quality of the work you do be relatively high. Success can mean doing the problem correctly or failing to do the problem correctly but displaying some understanding of where things go wrong.  This later form of success is likely to require a bit of meta-cognitive analysis and a sentence of two discussing where and why you got stuck. High Quality work should be neat, organized and make good use of both prose and mathematical formalism.  If you yourself do not find the act of following your work easy and enjoyable, then it's probably not of high quality.


\end{document}
