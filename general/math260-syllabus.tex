\documentclass[nobib]{tufte-handout}
\usepackage{amsmath}

\title{Syllabus \\ MATH 260 \\ Discrete Mathematics}
\author{}
\date{ Fall 2017 }

\begin{document}
\maketitle

\section{Logistics}
\begin{itemize}
\item \textbf{Where: } Center for Science and Business, Room 309
\item \textbf{When: } MTWF 11:00--11:50 am
\item \textbf{Instructor:} James \textit{Logan} Mayfield
\begin{itemize}
\item \textit{Office: } Center for Science and Business, Room 344
\item \textit{Phone: } 309-457-2200 %chktex 8
\item \textit{Email: } lmayfield \textit{at} monmouthcollege \textit{dot} edu
\item \textit{Office Hours: } By Appointment*
\end{itemize}
\item \textbf{Website: } \url{https://jlmayfield.github.io/teaching/MATH260/}
\item \textbf{Credits: } 1 course credit
\end{itemize}

\section{Overview}


\section{Textbook}



\section{Assignments}


\section{Grading}

This courses uses a standard grading scale.  Assignments and final grades will not be curved except in rare cases when it's deemed necessary by the instruction.  Percentage grades translate to letter grades as follows:
\newline
\begin{center}
\begin{small}
\begin{tabular}{ll}
\underline{Score Range} & \underline{Grade} \\
94--100 & A \\
90--93 & A- \\
88--89 & B+ \\
82--87 & B \\
80--81 & B- \\
78--79 & C+ \\
72--77 & C \\
70--71 & C- \\
68--69 & D+ \\
62--67 & D \\
60--61 & D- \\
0--59 & F
\end{tabular}
\end{small}
\end{center}


Students are always welcome to challenge a grade that they feel is unfair or calculated incorrectly.  Mistakes made in the student's favor will never be corrected to lower a grade.  Mistakes made not in a student's favor will be corrected.  \textit{Basically, after the initial grading of an assignment, it's score can only go up as the result of a challenge.}


\subsection{Grade Weights}

The final grade is based on a weighted average of particular assignment categories.  Students may visit the instructor \textit{outside of class time} to discuss their current standing.

\begin{center}
\begin{tabular}{ll}
\underline{Category} & \underline{Weight}  \\
Participation \& Attendance & \% \\
Problem Sets & \% \\
Exams & \% \\
\end{tabular}
\end{center}


\section{Calendar}

\textit{This calendar is subject to change based on the circumstances of the course.}  The \textit{Journals Remaining} column lists how many more new entries for the current half-semester can be submitted after that week's submission.

\begin{center}
\begin{tabular}{lll}

\underline{Week} & \underline{Dates} & \underline{Assignments} \\
1 & 8/22 --- 8/25 &  \\
2 & 8/28 --- 9/1 &  \\
3 & 9/4 --- 9/8 &  \\
4 & 9/11 --- 9/15 &  \\
5 & 9/18 --- 9/22 &  \\
6 & 9/25 --- 9/29 &  \\
7 & 10/2 --- 10/6 &    \\
8 & 10/9 --- 10/10 &  FALL BREAK (WThF)\\
9 & 10/16 --- 10/20 &  \\
10 & 10/23 --- 10/27 &  \\
11 & 10/30 --- 11/3 &  \\
12 & 11/6 --- 11/10 &  \\
13 & 11/13 --- 11/17 &  \\
14 & 11/20 --- 11/21 &  THANKSGIVING BREAK (WThF) \\
15 & 11/27 --- 12/1 &  \\
16 & 12/4 --- 12/6 &  READING DAY (Th) \\
Final's Week & 5/11 (8:00--11:00am) &  \\
\end{tabular}
\end{center}

\subsection{Course Engagement Expectations}

The weekly workload for this course will vary by student but on average should be about 12--13 hours per week.  The follow tables provides a rough estimate of the distribution of this time over different course components for a 15 week semester.
\begin{center}
\begin{tabular}{lll}
In-Class &      & 3 hours/week \\
Finding and working with resources &        & 4 hours/week \\
Journal &   & 2.5 hours/week \\
Papers & 48 hours & 3.2 hours/week \\
& & 12.7 hours/week \\
\end{tabular}
\end{center}


\end{document}
