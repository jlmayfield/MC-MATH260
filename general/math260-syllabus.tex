\documentclass[nobib]{tufte-handout}
\usepackage{amsmath}

\title{Syllabus \\ MATH 260 --- Discrete Mathematics}
\author{}
\date{ Fall 2017 }

\begin{document}
\maketitle

\section{Logistics}
\begin{itemize}
\item \textbf{Where: } Center for Science and Business, Room 309
\item \textbf{When: } MTWF 11:00--11:50 am
\item \textbf{Instructor:} James \textit{Logan} Mayfield
\begin{itemize}
\item \textit{Office: } Center for Science and Business, Room 344
\item \textit{Phone: } 309-457-2200 %chktex 8
\item \textit{Email: } lmayfield \textit{at} monmouthcollege \textit{dot} edu
\item \textit{Website: } \url{https://jlmayfield.github.io/}
\item \textit{Office Hours: } Tuesday \& Thursday 9-10am. Friday 1-2pm. By Appointment.
\end{itemize}
\item \textbf{Website: } \url{https://jlmayfield.github.io/teaching/MATH260/}
\item \textbf{Credits: } 1 course credit
\end{itemize}

\section{Overview}

This course introduces students to the core topics in Discrete Mathematics while simultaneously developing their skills in mathematical reasoning and proof-based mathematics. In doing so it serves as an introduction to higher-level mathematics and more advanced coursework. Students will learn to express and work with precise mathematical definitions in order to  conceive of and ultimately prove or disprove new mathematical theorems.  A strong emphasis is placed on developing and honing the skills necessary to communicate new results in mathematics to a wider community.  In short, students will learn how to think and work as a mathematician through a close examination of Discrete Mathematics.

Topics covered in this course include the the core definitions and theorems from set theory, mathematical logic, and combinatorics.  In this context students will explore ideas such as mathematical induction, functions, and relations and the proof techniques of direct proof, contrapositive proof, proof by contradiction, and disproof.


\section{Textbook}

\noindent
Hammack, Richard\@. \textit{Book of Proof}. Second Edition. Licensed under CC BY ND. Available at \url{http://www.people.vcu.edu/~rhammack/BookOfProof/}. ISBN: 978-0-9894721-0-4. % chktex 8

\section{Assignments and Workload}

This course is broken down into seven parts. Each part has a associated set of problems and an exam. Class time will occasionally be used to work on problem sets but the bulk of the work for these sets is to be done outside of class.  Problems will run the gambit from doing basic calculations to writing and communicating proofs.

\subsection{Course Engagement Expectations}

The weekly workload for this course will vary by student but on average should be about 12 hours per week.  The follow table provides a rough estimate of the distribution of this time over different course components for a 15 week semester.
\begin{center}
\begin{tabular}{lll}
In-Class &      & 4 hours/week \\
Problem Sets &  & 5 hours/week \\
Exam Prep &   & 3 hours/week \\
\end{tabular}
\end{center}


\section{Grading}

This course uses a standard grading scale.  Grades will not be curved except in rare cases when it's deemed necessary by the instructor.  Percentage grades translate to letter grades as follows:
\newline
\begin{center}
\begin{small}
\begin{tabular}{ll}
\underline{Score Range} & \underline{Grade} \\
94--100 & A \\
90--93 & A- \\
88--89 & B+ \\
82--87 & B \\
80--81 & B- \\
78--79 & C+ \\
72--77 & C \\
70--71 & C- \\
68--69 & D+ \\
62--67 & D \\
60--61 & D- \\
0--59 & F
\end{tabular}
\end{small}
\end{center}


Students are always welcome to challenge a grade that they feel is unfair or calculated incorrectly.  Mistakes made in the student's favor will never be corrected to lower a grade.  Mistakes made not in a student's favor will be corrected.  \textit{Basically, after the initial grading of an assignment, it's score can only go up as the result of a grade challenge.}


\subsection{Grade Weights}

The final grade is based on a weighted average.  Students may visit the instructor \textit{outside of class time} to discuss their current standing in the course.

Each problem set carries the same weight in terms of the final course grade. Problem sets are graded on a simple 3 point scale with an emphasis on trying the right kinds of things for each problem while mostly achieving success. It's not about being right, it's engaging in the problem in a reasoned and thoughtful manner and demonstrating that through your work.  All exams carry the same weight and will be graded with an emphasis on both correctness and approach.

Students are expected to be in class and to regularly engage in the act of exploring and creating mathematics. Unexcused absences from class and repeated failures to show-up to class prepared to engage in the material will result in a precipitous drop in the participation component of the final grade.

\begin{center}
\begin{tabular}{ll}
\underline{Category} & \underline{Weight}  \\
Participation \& Attendance & 10\% \\
Problem Sets & 35\% \\
Exams & 55\%
\end{tabular}
\end{center}

\subsubsection{Problem Set Grading}

Problem sets will be graded with a simple 3 point scale. A more exact rubric can be found on the course website and will be attached to the actual assignments. Your final problem set grade is determined by your average problem set score and drawn from the table below. Notice this chart lists the minimum average needed to achieve a particular letter grade.

\begin{center}
\begin{small}
\begin{tabular}{cc}
\underline{Assignment Avg. (Min)} & \underline{Letter Grade} \\
2.8   & A  \\
2.75    & A- \\
2.5 & B+ \\
2.25    & B  \\
2   & B- \\
1.75    & C+ \\
1.5 & C  \\
1   & C- \\
0.75    & D  \\
0.5  & F
\end{tabular}
\end{small}
\end{center}

\section{Calendar}

\textit{This calendar is subject to change based on the circumstances of the course.}

\begin{center}
\begin{tabular}{llll}
\underline{Week} & \underline{Dates} & \underline{Assignments Due} & \underline{Chapter(s)}\\
1 & 8/22 --- 8/25 & & 1.1--2 \\
2 & 8/28 --- 9/1 & & 1.3--1.8 \\
3 & 9/4 --- 9/8 & PSet 1. (Tu) & 2.1--4 \\
4 & 9/11 --- 9/15 & Exam 1 (M) & 2.5--7 \\
5 & 9/18 --- 9/22 & PSet 2. (W) & 3.1--3 \\
6 & 9/25 --- 9/29 & Exam 2 (Tu). & 3.4--5 \\
7 & 10/2 --- 10/6 &  PSet 3. (Tu) & 4,5.1--2  \\
8 & 10/9 --- 10/10 & Exam 3 (M). FALL BREAK (WThF) &  6.1\\
9 & 10/16 --- 10/20 & PSet 4. (F) & 6.1--3. 7.1--3. \\
10 & 10/23 --- 10/27 & Exam 4 (Tu) & 8.1--4 \\
11 & 10/30 --- 11/3 &  PSet 5 (Tu). & 9,10.0 \\
12 & 11/6 --- 11/10 &  Exam 5 (M). PSet 6(F) & 10.1--2,11.0--1 \\
13 & 11/13 --- 11/17 & Exam 6 (Tu) & 11.1--5 \\
14 & 11/20 --- 11/21 &  THANKSGIVING BREAK (WThF) & 12.1 \\
15 & 11/27 --- 12/1 &  & 12.2--5\\
16 & 12/4 --- 12/6 & PSet 7 (Tu).  READING DAY (Th) & 12.5--6\\
Final's Week & 12/11 (8:00--11:00am) & Exam 7. &  \\
\end{tabular}
\end{center}


\end{document}
